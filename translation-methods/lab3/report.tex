\documentclass{article}

\usepackage[utf8]{inputenc} % если ваш файл содержит русский текст, нужно указать кодировку
\usepackage[russian]{babel} % для того, чтобы писать русский текст
\usepackage{amsmath} % для команды equation*
\usepackage{hyperref} % для вставки ссылок
\usepackage{graphicx}
\usepackage{amsfonts}
\usepackage{dsfont}
\usepackage{textcomp}
\usepackage{multicol}
\usepackage{verbatim}
\usepackage{makecell}
\usepackage{amssymb}
\title{Лабораторная работа №3. Использование автоматических генераторов анализаторов Bison и ANTLR}
\date{22 ноября 2021 г.}
\author{Наумов Иван M33391 \\ Вариант 2\\ Арифметические выражения}

%\usepackage{vmargin}
%\setpapersize{A4}
%\setmarginsrb{2cm}{1.5cm}{1cm}{1.5cm}{0pt}{0mm}{0pt}{13mm}
%\usepackage{indentfirst}
%\sloppy
\begin{document}
	\maketitle
	
	\section*{Грамматика}
	Воспользуемся следующей грамматикой для разбора арифметических выражений со стандартными операциями, скобками и унарным минусом:
	\begin{multicols}{2}
		\begin{align}
			E & \rightarrow E+T \\
			E & \rightarrow E-T \\
			E & \rightarrow T \\
			T & \rightarrow T*F \\
			T & \rightarrow T/F \\
			T & \rightarrow F \\
			F & \rightarrow (E) \\
			F & \rightarrow -F \\
			F & \rightarrow n
		\end{align}
		
		\columnbreak
	
		\begin{align*}
			\begin{tabular}{|c|c|}
			\hline
			$E$ & Выражение \\
			\hline
			$T$ & Слагаемое(Вычитаемое) \\
			\hline
			$F$ & Множитель(Делитель) \\
			\hline
			$n$ & Неотриц. число \\
			\hline
			\end{tabular}
		\end{align*}
	\end{multicols}

	\newpage

	Избавимся от левой рекурсии:
	
	\begin{multicols}{2}
		\setcounter{equation}{0}
		\begin{align}
			E & \rightarrow TE' \\
			E' & \rightarrow +TE' \\
			E' & \rightarrow -TE' \\
			E' & \rightarrow \varepsilon \\
			T & \rightarrow FT' \\
			T' & \rightarrow *FT' \\
			T' & \rightarrow /FT' \\
			T' & \rightarrow \varepsilon \\
			F & \rightarrow (E) \\
			F & \rightarrow -F \\
			F & \rightarrow n
		\end{align}
		
		\columnbreak
		
		\begin{align*}
			\begin{tabular}{|c|c|}
			\hline
			$E$ & Выражение \\
			\hline
			$E'$ & Правая часть сложения/вычитания \\
			\hline
			$T$ & Слагаемое(Вычитаемое) \\
			\hline
			$T'$ & Правая часть умножения/деления \\
			\hline
			$F$ & Множитель(Делитель) \\
			\hline
			$n$ & Неотриц. число \\
			\hline
			\end{tabular}
		\end{align*}
	\end{multicols}

	\section*{Реализация}
	
	Воспользуемся автоматическим генератором парсеров ANTLR. Заведем для каждого нетерминала синтезируемый атрибут для значения этого выражения. Для $E'$ и $T'$ заведем также и наследуемый - значение левой части. Также заведем глобальный словарь $\operatorname{memory}$ для хранения значений переменных и StringBuilder $\operatorname{sb}$ для вывода.
	
	\verbatiminput{src/main/grammar/Expr.g4}
	
\end{document}