\documentclass{article}

\usepackage[utf8]{inputenc} % если ваш файл содержит русский текст, нужно указать кодировку
\usepackage[russian]{babel} % для того, чтобы писать русский текст
\usepackage{amsmath} % для команды equation*
\usepackage{hyperref} % для вставки ссылок
\usepackage{graphicx}
\usepackage{amsfonts}
\usepackage{dsfont}
\usepackage{textcomp}
\usepackage{multicol}
\usepackage{verbatim}
\usepackage{makecell}
\usepackage{amssymb}
\title{Лабораторная работа №2. Ручное построение нисходящих синтаксических анализаторов}
\date{15 ноября 2021 г.}
\author{Наумов Иван M33391 \\ Вариант 1. Арифметические выражения}

%\usepackage{vmargin}
%\setpapersize{A4}
%\setmarginsrb{2cm}{1.5cm}{1cm}{1.5cm}{0pt}{0mm}{0pt}{13mm}
%\usepackage{indentfirst}
%\sloppy
\begin{document}
	\maketitle
	
	\section{Разработка грамматики}
	\subsection*{Изначальная грамматика}
	
	\begin{multicols}{2}
		\begin{align}
			E & \rightarrow E+T \\
			E & \rightarrow E-T \\
			E & \rightarrow T \\
			T & \rightarrow T*F \\
			T & \rightarrow T/F \\
			T & \rightarrow F \\
			F & \rightarrow f(E) \\
			F & \rightarrow (E) \\
			F & \rightarrow -F \\
			F & \rightarrow n
		\end{align}
		
		\columnbreak
		
		\begin{align*}
			\begin{tabular}{|c|c|}
				\hline
				$E$ & Выражение \\
				\hline
				$T$ & Слагаемое(Вычитаемое) \\
				\hline
				$F$ & Множитель(Делитель) \\
				\hline
				$f$ & Функция \\
				\hline
				$n$ & Неотриц. число \\
				\hline
			\end{tabular}
		\end{align*}
	\end{multicols}
	Правила 1, 2, 4, 5 содержат левую рекурсию, избавимся от нее.

	\newpage

	\begin{multicols}{2}
		\setcounter{equation}{0}
		\begin{align}
			E & \rightarrow TE' \\
			E' & \rightarrow +TE' \\
			E' & \rightarrow -TE' \\
			E' & \rightarrow \varepsilon \\
			T & \rightarrow FT' \\
			T' & \rightarrow *FT' \\
			T' & \rightarrow /FT' \\
			T' & \rightarrow \varepsilon \\
			F & \rightarrow f(E) \\
			F & \rightarrow (E) \\
			F & \rightarrow -F \\
			F & \rightarrow n
		\end{align}
		
		\columnbreak
		
		\begin{align*}
			\begin{tabular}{|c|c|}
				\hline
				$E$ & Выражение \\
				\hline
				$E'$ & Правая часть сложения/вычитания \\
				\hline
				$T$ & Слагаемое(Вычитаемое) \\
				\hline
				$T'$ & Правая часть умножения/деления \\
				\hline
				$F$ & Множитель(Делитель) \\
				\hline
				$f$ & Функция \\
				\hline
				$n$ & Неотриц. число \\
				\hline
			\end{tabular}
		\end{align*}
	\end{multicols}

	\section{Построение лексического анализатора}
	В получившейся грамматике 9 терминалов:
	\begin{center}
		\begin{tabular}{|c|c|}
			\hline
			Терминал & Токен \\
			\hline
			f & FUN \\
			\hline
			+ & PLUS \\
			\hline
			- & MINUS \\
			\hline
			* & ASTER \\
			\hline
			/ & SLASH \\
			\hline
			( & LPAREN \\
			\hline
			) & RPAREN \\
			\hline
			n & NUMBER \\
			\hline
			\$ & END \\
			\hline
		\end{tabular}
	\end{center}
	Построим лексический анализотор. Заведем также токен для обозначения ошибки разбора.
	\subsection*{lexer.h}
	\verbatiminput{lexer.h}
	\subsection*{lexer.cpp}
	\verbatiminput{lexer.cpp}
	
	\section{Построение синтаксического анализатора}
	Построим множества FIRST и FOLLOW для нетерминалов нашей грамматики.
	\begin{center}
		\begin{tabular}{|c|c|c|}
			\hline
			Нетерминал & FIRST & FOLLOW \\
			\hline
			$E$ & ${f, (, -, n}$ & ${\$, )}$ \\
			\hline
			$E'$ & ${+, -, \varepsilon}$ & ${\$, )}$ \\
			\hline
			$T$ & ${f, (, -, n}$ & ${+, - \$, )}$ \\
			\hline
			$T'$ & ${*, /, \varepsilon}$ & ${+, -, \$, )}$ \\
			\hline
			$F$ & ${f, (, -, n}$ & ${*, /, +, -, \$, )}$ \\
			\hline
		\end{tabular}
	\end{center}
	Проверим, что наша грамматика принадлежит классу LL(1):
	\begin{center}
		\begin{tabular}{|c|c|c|c|}
			\hline
			& $A \rightarrow \alpha$ & $FIRST(\alpha)$ & $FOLLOW[A]$ \\
			\hline
			OK & $E \rightarrow TE'$ & (, $-$, $f$, $n$ & \$, ) \\
			\hline
			OK &
			  \makecell{$E' \rightarrow +TE'$ \\ $E' \rightarrow -TE'$ \\ $E' \rightarrow \varepsilon$} &
			  \makecell{$+$ \\ $-$ \\ $\varepsilon$} &
			  \$, ) \\
			\hline
			OK & $T \rightarrow FT'$ & (, $-$, $f$, $n$ & \$, ), $+$, $-$ \\
			\hline
			OK &
			  \makecell{$T' \rightarrow *FT'$ \\ $T' \rightarrow /FT'$ \\ $T' \rightarrow \varepsilon$} &
			  \makecell{$*$ \\ $/$ \\ $\varepsilon$} &
			  \$, ), $+$, $-$ \\
			\hline
			OK &
			  \makecell{$F \rightarrow f(E)$ \\ $F \rightarrow (E)$ \\ $F \rightarrow -F$ \\ $F \rightarrow n$} &
			  \makecell{$f$ \\ ( \\ $-$ \\ $n$} &
			  \$, ), $+$, $-$, $*$, $/$ \\
			\hline
		\end{tabular}
	\end{center}

	$\forall A \rightarrow \alpha, A \rightarrow \beta$ \\
	1) \; $\operatorname{FIRST}(\alpha) \cap\operatorname{FIRST}(\beta) = \varnothing$ \\
	2) \; $\varepsilon \in \operatorname{FIRST}(\alpha) \implies \operatorname{FIRST}(\beta) \cap \operatorname{FOLLOW}[A] = \varnothing$ \\
	Составленная грамматика удовлетворяет условию, поэтому является LL(1), соответственно, для нее можно написать нисходящий парсер.
	
	\subsection*{parse.cpp}
	\verbatiminput{parse.cpp}
	
	\section{Визуализация дерева разбора}
	
	Сгенерированное изображение для дерева разбора выражения $5 * -(2 + 3)$:\\
	\includegraphics*[scale=0.4]{cmake-build-debug/graph.png}
	
	\section*{Модификация: функции произвольного числа аргументов}
	
	Модифицируем грамматику:
	
	\begin{multicols}{2}
		\setcounter{equation}{0}
		\begin{align}
			E & \rightarrow TE' \\
			E' & \rightarrow +TE' \\
			E' & \rightarrow -TE' \\
			E' & \rightarrow \varepsilon \\
			T & \rightarrow FT' \\
			T' & \rightarrow *FT' \\
			T' & \rightarrow /FT' \\
			T' & \rightarrow \varepsilon \\
			F & \rightarrow f(A) \\
			F & \rightarrow (E) \\
			F & \rightarrow -F \\
			F & \rightarrow n \\
			A & \rightarrow EA' \\
			A & \rightarrow \varepsilon \\
			A' & \rightarrow ,EA' \\
			A' & \rightarrow \varepsilon
		\end{align}
		
		\columnbreak
		
		\begin{align*}
			\begin{tabular}{|c|c|}
			\hline
			$E$ & Выражение \\
			\hline
			$E'$ & Правая часть сложения/вычитания \\
			\hline
			$T$ & Слагаемое(Вычитаемое) \\
			\hline
			$T'$ & Правая часть умножения/деления \\
			\hline
			$F$ & Множитель(Делитель) \\
			\hline
			$A$ & Аргументы функции \\
			\hline
			$A'$ & Запятая + остаток аргументов функции \\
			\hline
			$f$ & Функция \\
			\hline
			$n$ & Неотриц. число \\
			\hline
			\end{tabular}
		\end{align*}
	\end{multicols}
	
	Проверим, что наша грамматика принадлежит классу LL(1):
	\begin{center}
		\begin{tabular}{|c|c|c|c|}
			\hline
			& $A \rightarrow \alpha$ & $FIRST(\alpha)$ & $FOLLOW[A]$ \\
			\hline
			OK & $E \rightarrow TE'$ & ( $-$ $f$ $n$ & \$ ) , \\
			\hline
			OK &
			\makecell{$E' \rightarrow +TE'$ \\ $E' \rightarrow -TE'$ \\ $E' \rightarrow \varepsilon$} &
			\makecell{$+$ \\ $-$ \\ $\varepsilon$} &
			\$ ) , \\
			\hline
			OK & $T \rightarrow FT'$ & ( $-$ $f$ $n$ & \$ ) $+$ $-$ , \\
			\hline
			OK &
			\makecell{$T' \rightarrow *FT'$ \\ $T' \rightarrow /FT'$ \\ $T' \rightarrow \varepsilon$} &
			\makecell{$*$ \\ $/$ \\ $\varepsilon$} &
			\$ ) $+$ $-$ , \\
			\hline
			OK &
			\makecell{$F \rightarrow f(A)$ \\ $F \rightarrow (E)$ \\ $F \rightarrow -F$ \\ $F \rightarrow n$} &
			\makecell{$f$ \\ ( \\ $-$ \\ $n$} &
			\$, ) $+$ $-$ $*$ $/$ , \\
			\hline
			OK &
			\makecell{$A \rightarrow EA'$ \\ $A \rightarrow \varepsilon$} &
			\makecell{( $-$ $f$ $n$ \\ $\varepsilon$} &
			)  \\
			\hline
			OK &
			\makecell{$A' \rightarrow ,EA'$ \\ $A' \rightarrow \varepsilon$} &
			\makecell{, \\ $\varepsilon$} &
			)  \\
			\hline
		\end{tabular}
	\end{center}

\end{document}